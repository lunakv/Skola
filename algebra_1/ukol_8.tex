\documentclass{scrartcl}
\usepackage[utf8]{inputenc}
\usepackage[czech]{babel}
\usepackage{amsmath}
\usepackage{amsfonts}
\usepackage{scrlayer-scrpage}

\cohead[Algebra I]
        {Algebra I}
\lohead[Úkol č. 8]
        {Úkol č. 8}
\rohead[Václav Luňák]
        {Václav Luňák}
\pagestyle{plain.scrheadings}

\begin{document}
    \section*{Úloha č. 2}
    \subsection*{a)}
    Jelikož 7 je prvočíslo, víme, že každý nenulový prvek $\mathbb{Z}_7$ je generátor grupy $(\mathbb{Z}_7,+)$. Z toho vyplývá, že pokud je v nějaké podmnožině $\mathbb{Z}_7$ nenulový prvek a požadujeme uzavřenost na sčítání, musí množina obsahovat celou $\mathbb{Z}_7$.\\

    Z předchozího tedy můžeme snadno nahlédnout, že podalgebry $\mathbb{Z}_7$ (+) jsou právě množiny $\emptyset$, $\{0\}$ a $\mathbb{Z}_7$.

    \subsection*{b)}
    Nejprve si uvědomme, že přítomnost 0 nemá vliv na uzavřenost množiny vzhledem k násobení ($0 \cdot x = 0$ pro všechna $x$). Vidíme tedy, že množina $P$ je podalgebrou $\mathbb{Z}_7$ ($\cdot$) právě tehdy, když je podalgebrou množina $P \cup \{0\}$, resp. $P \setminus \{0\}$. Dále tedy BÚNO nebudeme přítomnost nuly uvažovat.\\

    Prvky 3 a 5 jsou generátory grupy $(\mathbb{Z}_7 \setminus \{0\}, \cdot)$. Stejně jako v předchozím případě tedy množiny obsahující 3 nebo 5 musí obsahovat celou $\mathbb{Z}_7 \setminus \{0\}$. 
    Ostatní prvky generují následující množiny:
    \begin{itemize}
        \item $\langle 1 \rangle = \{1\}$
        \item $\langle 2 \rangle = \langle 4 \rangle = \{2,4,1\}$
        \item $\langle 6 \rangle = \{1,6\}$
    \end{itemize}
    Tyto množiny jsou triviálně podalgebry. Jelikož $2 \cdot 6 = 5$ a $4 \cdot 6 = 3$, množiny obsahující tyto dvojice prvků opět musí obsahovat celou množinu. Tím jsme ukázali, že žádné další podalgebry nemohou existovat. \\

    Podalgebrami $\mathbb{Z}_7$ $(\cdot)$ jsou tedy množiny $\emptyset$, $\{1\}$, $\{1,2,4\}$, $\{1,6\}$, $\mathbb{Z}_7 \setminus \{0\}$ a jejich sjednocení s $\{0\}$

\end{document}