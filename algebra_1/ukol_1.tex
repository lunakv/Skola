\documentclass{scrartcl}
\usepackage[utf8]{inputenc}
\usepackage{scrlayer-scrpage}
\usepackage[czech]{babel}
\usepackage{amsmath}
\usepackage{amsfonts}

\cohead[Algebra I]
        {Algebra I}
\lohead[Úkol č. 1]
        {Úkol č. 1}
\rohead[Václav Luňák]
        {Václav Luňák}
\pagestyle{plain.scrheadings}

\begin{document}
\subsection*{Úloha č. 1}
\paragraph{a)} Vlastnosti ekvivalence plynou z rovnosti v definici
\begin{description}
        \item[reflexivita] $L(\textbf{x}) = L(\textbf{x}) \implies (\textbf{x}, \textbf{x}) \in \text{ker } L $
        \item[symetrie] $(\textbf{x}, \textbf{y}) \in \text{ker } L \implies L(\textbf{x}) = L(\textbf{y}) \implies L(\textbf{y}) = L(\textbf{x}) \implies (\textbf{y}, \textbf{x}) \in \text{ker } L$
        \item[tranzitivita] $L(\textbf{x}) = L(\textbf{y}) \land L(\textbf{y}) = L(\textbf{z}) \implies L(\textbf{x}) = L(\textbf{z}) \implies (\textbf{x}, \textbf{z}) \in \text{ker } L$  
\end{description}

\paragraph{b)}
Mějme bod $\textbf{x} = (x,y,z)$ splňující $x+y+z = n$. Potom $[\textbf{x}]$ je rovina obsahují všechny body splňující tuto rovnici. (Promítnutím této roviny do některé z kanonických rovin vznikne přímka s gradientem -1). Správnost tohoto rozkladu plyne přímo z definice $L$.
\paragraph{c)}
$T = \{(x,0,0) \vert x \in \mathbb{R}\}$.\\
Že dva různé prvky $T$ jsou v různé třídě ekvivalence vidíme triviálně (mají jiné součty souřadnic). Vezmeme-li $n$ z rovnice pro nějakou třídu ekvivalence, bod $(n,0,0)$ náleží $T$ i této třídě. $T$ tedy pokrývá všechny třídy ekvivalence, čímž splňujeme i první podmínku.

\subsection*{Úloha č. 2}
\begin{align*}
        & 34x + 21y = 8 \\
        & \text{NSD}(34,21) = 1 \\
        & 34 \\
        & 21 \\
        & 13 = 34 - 21 \\
        & 8 = 21 - 13 = -34 + 2 \cdot 21
\end{align*}
Řešením je tedy množina $\{(-1,2) + k(21,-34) \vert k \in \mathbb{Z}\}$. Platnost postupu viz. cvičení.

\subsection*{Úloha č. 3}
Rovnice soustavy jsou nezávislé, tedy pro každý racionální parametr bude množina řešení neprázdná. Nechť má soustava pro nějaký racionální parametr $(a,b,c)$ řešení $(x_0,y_0,z_0)$. Aby byla množina řešení uzavřená na součty, musí obsahovat i vektor $2(x_0,y_0,z_0)$. \\

Zároveň ovšem snadno vidíme, že po dosazení $2(x_0,y_0,z_0)$ do soustavy dostaneme pravou stranu rovnou $2(a,b,c)$. Protože je $2(x_0,y_0,z_0)$ řešením soustavy, musí platit $2(a,b,c) = (a,b,c) \implies (a,b,c) = 0$.

\end{document}