\documentclass{scrartcl}
\usepackage[utf8]{inputenc}
\usepackage[czech]{babel}
\usepackage{amsmath}
\usepackage{amsfonts}
\usepackage{scrlayer-scrpage}

\cohead[Algebra I]
        {Algebra I}
\lohead[Úkol č. 7]
        {Úkol č. 7}
\rohead[Václav Luňák]
        {Václav Luňák}
\pagestyle{plain.scrheadings}

\begin{document}
    \section*{Úloha č. 1}
    Snadno ukážeme, že $T(A)$ je grupa. Z komutativnosti $A$ je to pak triviálně normální podgrupa, tudíž můžeme definovat faktorovou grupu $A/T(A)$. Z definice je neutrální prvek této grupy $e \cdot T(A) = T(A)$ pro $e$ neutrální prvek $A$.\\

    Předpokládejme, že pro nějaké $a \in A$ je $a \cdot T(A)$ konečného řádu. Potom musí platit pro nějaké přirozené $k,n$
    \begin{align*}
        (a\cdot T(A))^k = T(A) \implies
        (a^k) \cdot T(A) = T(A) \implies
        (\forall x \in T(A))(a^k \cdot x \in T(A)) \implies \\
        a^k \cdot e \in T(A) \implies a^k \in T(A) \implies
        {(a^k)}^n = e \implies a^{kn} = e \implies 
        a \in T(A) \implies \\ a\cdot T(A) = T(A),
    \end{align*}
    kde poslední implikace plyne z uzavřenosti $T(A)$. Z tohoto pak vyplývá, že jediný prvek v $A/T(A)$ s konečným řádem je ten neutrální.
    
    \section*{Úloha č. 2}
    Mějme prvočíslo $p$ a přirozené $k$. Vezměme z faktorové grupy prvek $\frac{1}{p^k} + \mathbb{Z}$. Pro tento prvek platí
    \begin{align*}
        p^k\left(\frac{1}{p^k} + \mathbb{Z}\right) = p^k\frac{1}{p^k} + \mathbb{Z} = 1 + \mathbb{Z} = \mathbb{Z},
    \end{align*}
    tudíž řád tohoto prvku je nejvýše $p^k$. Jelikož pro libovolné menší $n$ není $\frac{n}{p^k}$ celé číslo, je jeho řád roven $p^k$.

    \section*{Úloha č. 3}
    \subsection*{a)}
    Využijeme vlastností homomorfismu.
    \subsubsection*{Uzavřenost}
    \begin{align*}
        f_0(g_1)(x) = x \land f_0(g_2)(x) = x \implies f_0(g_1 \cdot g_2)(x) = (f_0(g_1) \circ f_0(g_2))(x) = x,
    \end{align*}.
    \subsubsection*{Neutrální prvek}
    Pro neutrální prvek $e \in G$ musí platit $f_0(e) = id$, tudíž $e$ je stabilizátor $x$. 
    \subsubsection*{Inverzní prvek}
    Pokud $f_0(g)(x) = x$, pak $f_0(g^{-1})(x) = f_0^{-1}(g)(x) = x$.

    \subsection*{b)}
    Nechť $h = \begin{pmatrix}
        1 & 2 \\
        0 & 2
    \end{pmatrix}$.
    \subsubsection*{Stabilizátor translace}
    Hledáme množinu matic $g$ takových, že $trans_g(h) = h \Leftrightarrow  g\cdot h = h$. 
    \begin{align*}
        \begin{pmatrix}
            a & b \\
            c & d
        \end{pmatrix} \cdot h &= 
        \begin{pmatrix}
            a & 2a+2b \\
            c & 2c+2d
        \end{pmatrix} = h \\
        a &= 1 \\
        c &= 0 \\
        2a+2b &= 2 \implies b = 0 \\
        2c+2d &= 2 \implies d = 1,
    \end{align*}
    z čehož vyplývá, že stabilizátorem $h$ při translaci je množina 
    \begin{align*}
        \left\{
            \begin{pmatrix}
                1 & 0 \\
                0 & 1
            \end{pmatrix}
        \right\}.
    \end{align*}
    \subsubsection*{Orbita transpozice}
    Matice náležící orbitě $h$ při transpozici jsou regulární matice vyjádřitelné jako 
    \begin{align*}
        \begin{pmatrix}
            a & b \\
            c & d \\
        \end{pmatrix} 
        \cdot h =
        \begin{pmatrix}
            a & 2a+2b \\
            c & 2c+2d
        \end{pmatrix}.
    \end{align*}
    Když vezmeme matice tohoto tvaru pro všechna $a,b,c,d \in \mathbb{Z}_3$, zjistíme, že dostáváme celou $GL_2(\mathbb{Z}_3)$.
\end{document}