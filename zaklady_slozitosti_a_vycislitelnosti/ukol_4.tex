\documentclass{scrartcl}
\usepackage[czech]{babel}
\usepackage{amsmath}
\usepackage{libertinus}
\usepackage{bm}
\usepackage{microtype}
\usepackage{scrlayer-scrpage}

\cohead[Základy složitosti a vyčíslitelnosti]{}
\lohead[Úkol č. 4]{}
\rohead[Václav Luňák]{}
\pagestyle{plain.scrheadings}

\DeclareMathOperator{\tm}{\text{TIME}}
\DeclareMathOperator{\s}{\text{SPACE}}
\DeclareMathOperator{\ntm}{\text{NTIME}}
\DeclareMathOperator{\ns}{\text{NSPACE}}

\begin{document}
    \begin{enumerate}
        \item $\s(n)\,?\,\tm(2^{\log^3 n})$. Žádný známý vztah nelze pro tento případ použít. Speciálně nemůžeme použít důsledek 1.3, jelikož $n \notin o(\log^3 n)$.
        \item $\tm(2^{\log^3 n}) \supset \ns(\log^2 n)$. 
        \begin{align*}
            \ns(\log^2 n) \subseteq \tm(2^{(\log n)^{5/2}}) \subset \tm(2^{\log^3 n}), 
        \end{align*}
        kde první inkluze plyne z důsledku 1.3 ($\log^2 n \in o((\log n)^{5/2}))$ a druhá z věty 1.6, jelikož $2^{\log^{5/2} n} \in o(2^{\log^3 n}/\log^{5/2} n)$.
        \item $\ns(\log^2 n) \subset \ntm(2^{\log^3 n})$. 
        \begin{align*}
            \ns(\log^2 n) \subset \tm(2^{\log^3 n}) \subseteq \ntm(2^{\log^3 n}).
        \end{align*}
        Předchozí příklad nám dává první inkluzi a druhá plyne z věty 1.1($ii$).
        \item $\ntm(2^{\log^3 n}) \subset \ntm(2^{n\log n})$ dostáváme z věty o nedeterministické časové hierarchii.
        \item $\ntm(2^{n\log n}) \supset \s(n)$.
        \begin{align*}
            \s(n) \subseteq \ns(n) \subseteq \tm(2^{n\log^{1/2} n}) \subset \tm(2^{n\log n}). 
        \end{align*}
        První inkluze plyne z věty 1.1($iii$), druhá z důsledku 1.3 a poslední z věty 1.6.
        \item $\s(n) \supset \ns(\log^2 n)$.
        \begin{align*}
            \ns(\log^2 n) \subseteq \s(\log^4 n) \subset \s(n),
        \end{align*}
        kde první inkluze plyne ze Savičovy věty a druhá z věty 1.5 ($\log^4 n \in o(n)$).
        \item $\tm(2^{\log^3 n}) \subseteq \ntm(2^{\log^3 n})$. Plyne přímo z věty 1.1($ii$).
        \item $\ns(\log^2 n) \subset \ntm(2^{n\log n})$.
        \begin{align*}
            \ns(\log^2 n) \subseteq \tm(2^{\log^3 n}) \subset \tm(2^{n\log n}) \subseteq \ntm(2^{n\log n}),
        \end{align*}
        což jsme získali použitím důsledku 1.3, věty 1.6 a věty 1.1($ii$).
        \item $\ntm(2^{\log^3 n})\, ?\, \s(n)$. Žádný známý vztah nelze použít, jelikož $2^{\log^3 n} \notin o(n)$, ale zároveň $2^n \notin o(2^{\log^3 n}/n)$.
        \item $\ntm(2^{n\log n}) \supset \tm(2^{\log^3 n})$.
        \begin{align*}
            \tm(2^{\log^3 n}) \subseteq \ntm(2^{\log^3 n}) \subset \ntm(2^{n\log n}),
        \end{align*}
        přičemž první inkluze vyplývá z věty 1.1($ii$) a druhou jsme ukázali ve čtvrtém příkladu.
    \end{enumerate}
\end{document}

