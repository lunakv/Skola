\documentclass[11pt,a4paper]{article}
\usepackage[utf8]{inputenc}
\usepackage[czech]{babel}
\usepackage[T1]{fontenc}
\usepackage{amsmath}
\usepackage{amsfonts}
\usepackage{amssymb}
\usepackage[includeheadfoot, margin=1in]{geometry}
\usepackage{fancyhdr}
\fancyhf{}
\pagestyle{fancy}
\rhead{Václav Luňák}
\lhead{Domácí úkol na 9. 5. 2018}
\cfoot{\thepage}
\begin{document}
\section*{Příklad 1}
Funkce je definovaná a spojitá na celé množině. Spočteme parciální derivace.
\begin{align*}
\frac{\partial f}{\partial x} = 2x - 6 \\
\frac{\partial f}{\partial y} = 2y - 4 
\end{align*}
Obě parciální derivace jsou definované a spojité na celé $M$. Z extrému je podezřelý bod $(3,2)$, neboť je v něm nulový gradient, a kraje množiny. Hassova matice pro bod $(3,2)$ je 
$$
\left(
\begin{matrix}
2 && 0\\
0 && 2
\end{matrix}
\right)
$$
Subdeterminanty této matice jsou všechny kladné, tedy v bodě $(3,2)$ je ostré lokální minimum. \\

K vyšetření funkce na okraji množiny ($\{(x,y) \vert x^2 + y^2 - 4x + 5 = 0\}$) použijeme Lagrangeovy multiplikátory.
\begin{align*}
L = x^2+y^2-6x-4y + 11 - \lambda (x^2+y^2-4x-5) \\
\end{align*}
\begin{align*}
\frac{\partial L}{\partial x} = && 2x - 6 - 2x\lambda + 4\lambda = 0 \\
\frac{\partial L}{\partial y} = && 2y - 4 - 2y\lambda = 0 \\
&& x^2+y^2-4x+5 = 0
\end{align*}
Po vyřešení této soustavy dostáváme dva extrémy na hranici množiny, $(2-\frac{3}{\sqrt{5}}, -\frac{6}{\sqrt{5}})$ a \\$(2+\frac{3}{\sqrt{5}}, \frac{6}{\sqrt{5}})$.
Porovnáním funkčních hodnot v těchto bodech a v bodě $(3,2)$ zjistíme, že globálního minima funkce nabývá v bodě $(3,2)$ a globálního maxima v bodě $(2-\frac{3}{\sqrt{5}}, -\frac{6}{\sqrt{5}})$. 
\section*{Příklad 2}
Funkce je definovaná a spojitá na celé množině. Jelikož $lim_{x\to \infty} e^{-x^2} = 0$ a tento výraz je vždy kladný, může $x$ nabývat všech hodnot v $R$, zatímco $y$ může nabývat hodnot na intervalu $[0,1)$. \\

Protože funkce není ve směru $x$ omezená a je to ve směru $x$ polynom druhého stupně, víme, že globální maximum je $+\infty$.\\

Minimum funkce spočteme přes Lagrangeovy multiplikátory.
\begin{equation*}
L = x^2 - y^2 - \lambda(y+e^{-x^2}-1)
\end{equation*}
\begin{align*}
\frac{\partial L}{\partial x} = && 2x -\lambda e^{-x^2} (-2x) = 0 \\
\frac{\partial L}{\partial y} = && -2y - \lambda = 0 \\
&& y+e^{-x^2}-1 = 0
\end{align*}
Vyřešením soustavy dostaneme jediné reálné řešení: bod $(0,0)$. V tomto bodě má rovnice globální minimum.

Pozn.: lze nahlédnout, že funkce je na celé množině nezáporná a nula tedy musí být globální minimum.
\begin{align*}
y+e^{-x^2}-1 = 0 \Rightarrow x^2 = \log(\frac{1}{1-y}) \\
h(y) = f(x, y) = \log(\frac{1}{1-y}) - y^2 > 0 \\
\log(\frac{1}{1-y}) > y^2 \\
\frac{1}{1-y} > e^{y^2} \\
1 > e^{y^2}(1-y)
\end{align*}
Jelikož $y$ je mezi 0 a 1, oba činitelé jsou < 1, tedy i součin je vždy < 1 a nerovnost platí.

\section*{Příklad 3}
Funkce je definovaná a spojitá na celém $R^2$. Zároveň je na celém definičním oboru nezáporná. Funkce je rovná nule pouze v bodě $(0,0)$, tudíž v tomto bodě musí být globální minimum. Pro nalezení globálního maxima nejprve spočteme parciální derivace
\begin{align*}
\frac{\partial f}{\partial x} = 2x\cdot e^{-5x^2-2y^2} + x^2\cdot e^{-5x^2-2y^2}\cdot (-10x) = xe^{-5x^2-2y^2} \cdot (2-10x^2)\\
\frac{\partial f}{\partial x} = 14y\cdot e^{-5x^2-2y^2} + 7y^2 \cdot e^{-5x^2-2y^2} \cdot (-4y) = 7ye^{-5x^2-2y^2} \cdot (2-4y^2)
\end{align*}
Body podezřelé z extrému jsou tedy:
$\{(x,y) \vert x \in \{0,\pm\frac{1}{\sqrt{5}}\}, y \in \{0, \pm\frac{1}{\sqrt{2}}\} \}$
Protože hledáme globální maximum funkce, zajímají nás z této množiny pouze body s nejvyšší funkční hodnotou, což jsou body $(0, \pm\frac{1}{\sqrt{2}})$.\\
Po dopočítání druhých derivací dostaneme pro bod $(0,\frac{1}{\sqrt{2}})$ Hessovu matici:
$$
\left(
\begin{matrix}
\frac{-33}{e} && 0 \\
0 && \frac{-28}{e}
\end{matrix}
\right),
$$
která je negativně definitní, tedy v tomto bodě je lokální maximum. Ze sudosti funkce musí tedy být lokální maximum i v bodě $(0, \frac{-1}{\sqrt{2}})$. Protože v nekonečnu se funkce blíží nule a toto jsou největší z možných lokálních maxim, jsou v těchto dvou bodech (neostrá) globální maxima.
\end{document}