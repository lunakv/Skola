\documentclass[11pt,a4paper]{article}
\usepackage[utf8]{inputenc}
\usepackage[czech]{babel}
\usepackage[T1]{fontenc}
\usepackage{amsmath}
\usepackage{amsfonts}
\usepackage{amssymb}
\usepackage[includeheadfoot, margin=1in]{geometry}
\usepackage{fancyhdr}
\fancyhf{}
\pagestyle{fancy}
\rhead{Václav Luňák}
\lhead{Domácí úkol na 10. 4. 2018}
\cfoot{\thepage}
\begin{document}
\section*{Příklad 1}
Vyjádříme inverzní funkci ve tvaru $x = y^3$, jelikož výchozí funkce je prostá. Objem se pak bude rovnat
\begin{equation*}
V = \pi \int_1^2 x^2 dy =\pi \int_1^2 y^6 dy =\pi \left[ \frac{y^7}{7} \right]_1^2 =
\pi\left(\frac{128}{7} - \frac{1}{7}\right) = \frac{127}{7}\pi
\end{equation*}

\section*{Příklad 2}
\begin{align*}
 & \left(\frac{1}{x}\right)' = \frac{-1}{x^2} 
 & \left(\left(\frac{1}{x}\right)'\right)^2 = \left(\frac{-1}{x^2}\right)^2 = \frac{1}{x^4} \\
\end{align*}
Dosadíme do vzorce pro povrch rotačního tělesa
\begin{equation*}
S = 2\pi \int_1^\infty \frac{1}{x} ds = 2\pi \int_1^\infty \frac{1}{x}\sqrt{1+\frac{1}{x^4}} dx = 2\pi \int_1^\infty \frac{1}{x^3} \sqrt{x^4+1} dx
\end{equation*}
$
\int_1^\infty \frac{\sqrt{x^4+1}}{x^3} dx 
$
konverguje právě tehdy, když pro $a_k = \frac{\sqrt{k^4+1}}{k^3}, k\in N$ konverguje $\sum_1^\infty a_k$. Jelikož $\sum_1^\infty a_k \geq \sum_1^\infty \frac{\sqrt{k^4}}{k^3} = \sum_1^\infty \frac{1}{k}$ a harmonická řada diverguje, ze srovnávacího kritéria diverguje i $\sum_1^\infty a_k$, tudíž z integrálního kritéria diverguje i $\int_1^\infty \frac{\sqrt{x^4+1}}{x^3} dx$, respektive jakýkoliv jeho reálný nenulový násobek. Protože integrovaný výraz je vždy kladný, ve výsledku dostáváme
\begin{equation*}
S = 2\pi \int_1^\infty \frac{1}{x^3} \sqrt{x^4+1} dx = \infty
\end{equation*}

\section*{Příklad 3}
\begin{equation*}
\int_\frac{1}{a}^a \frac{\log x }{x} dx =[subst.: t = \log x, dt = \frac{1}{x}dx]
= \int_{-\log a}^{\log a} t dt =
\left[ \frac{t^2}{2} \right]_{-\log a}^{\log a} = \frac{\log^2 a - (-\log)^2 a}{2} = 0
\end{equation*}

\end{document}