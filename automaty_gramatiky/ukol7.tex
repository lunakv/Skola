\documentclass{scrartcl}
\usepackage[utf8]{inputenc}
\usepackage{scrlayer-scrpage}
\usepackage[czech]{babel}
\usepackage{amsmath}
\usepackage{amssymb}

\cohead[Automaty a gramatiky]
        {Automaty a gramatiky}
\lohead[Úkol č. 7]
        {Úkol č. 7}
\rohead[Václav Luňák]
        {Václav Luňák}
\pagestyle{plain.scrheadings}

\begin{document} 
\textit{Terminály jsou značeny malými písmeny, neterminály velkými písmeny.}\\

Každá bezkontextová gramatika se dá převést na (až na $\lambda$) ekvivalentní gramatiku obsahující pouze pravidla typu
\begin{align*}
        A \rightarrow a \vert aB \vert aBC.
\end{align*}

\subsection*{Důkaz}
Mějme gramatiku v Greibachové normální formě, tedy gramatiku obsahující pouze pravidla typu 
\begin{align*}
        A \rightarrow a\beta, \beta \in V^*.
\end{align*}
Vezměme si jedno konkrétní pravidlo,
\begin{align*}
        A_i \rightarrow a_iA_{i_1}A_{i_2}\dots A_{i_n}\\ i, i_1 \dots i_n \in N.
\end{align*}
Předpokládejme, že $n \geq 2$. Toto pravidlo odstraníme a přidáme místo něj následující pravidla (s odpovídajícími novými neterminály):
\begin{align*}
        A_i &\rightarrow a_iX_{i_1i_2\dots i_n}\\
        X_{i_1i_2\dots i_n} &\rightarrow A_{i_1}X_{i_2i_3\dots i_n}\\
        X_{i_2i_3\dots i_n} &\rightarrow A_{i_2}X_{i_3i_4\dots i_n}\\
        &\vdots \\
        X_{i_{n-1}i_n} &\rightarrow A_{i_{n-1}}A_{i_n}
\end{align*}
Snadno vidíme, že tato substituce generuje ekvivalentní gramatiku, neboť pouze rozkládáme jednu substituci na několik jednoznačných substitucí. Tento proces opakujeme pro všechna pravidla v gramatice. (Těch je konečně mnoho, tedy výsledek je konečný.) Pozorujme, že všechna pravidla v této gramatice mají tvar
\begin{align*}
        A_i &\rightarrow a \vert aN \\
        X_\pi &\rightarrow A_iM
\end{align*}
Teď nám stačí pouze dosadit do pravidel pro $X$ za $A_i$ všechny jejich substituce, čímž ekvivalenci opět zachováme a navíc dostaneme pravidla ve tvaru
\begin{align*}
        A_i &\rightarrow a \vert aN \\
        X_\pi &\rightarrow aM \vert aNM,
\end{align*}
tedy všechna pravidla gramatiky mají požadovaný tvar.
\begin{flushright}
        $\square$
\end{flushright}
\end{document}
