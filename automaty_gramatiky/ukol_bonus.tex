\documentclass{scrartcl}
\usepackage[utf8]{inputenc}
\usepackage{scrlayer-scrpage}
\usepackage[czech]{babel}
\usepackage{amsmath}

\cohead[Automaty a gramatiky]
        {Automaty a gramatiky}
\lohead[Bonusový úkol]
        {Bonusový úkol}
\rohead[Václav Luňák]
        {Václav Luňák}
\pagestyle{plain.scrheadings}

\begin{document}
    Máme jazyk $L = \{\omega \in  \{a,b\}^* \text{ tž. } \vert \omega \vert_a\ = 2\vert \omega \vert_b\}$ a gramatiku $A$ se substitucemi
    \begin{align*}
        S \rightarrow \text{\uv{libovolná permutace řetězce $aabS$}} \vert \lambda.
    \end{align*} 
    Chceme ukázat, že $A$ negeneruje $L$.\\

    Mějme následující slovo
    \begin{align*}
        \omega = a^3b^3a^3.
    \end{align*}
    Ukážeme, že ačkoliv je toto slovo součástí jazyka $L$ (zjevně), nedá se generovat naší gramatikou.

    \paragraph{Důkaz}
    Předpokládejme pro spor, že máme nějakou posloupnost substitucí generující $\omega$. Vezměme první substituci této posloupnosti. Nechť se BÚNO v této substituci vyskytuje $b$ napravo od $S$ (v opačném případě postupuje důkaz symetricky). Napravo od tohoto $b$ mohou být po první substituci nejvýše dvě $a$. \\
    
    Zároveň vidíme, že v $\omega$ musí být napravo od posledního $b$ tři $a$. Jelikož se však jediný neterminál vyskytuje nalevo od $b$ a jsme v bezkontextové gramatice, nejsme schopni žádnou další substitucí přidat další znaky napravo od $b$. To nám dává spor s tím, že naše gramatika generuje $\omega$.
\end{document}