\documentclass[11pt,a4paper]{article}
\usepackage[utf8]{inputenc}
\usepackage[czech]{babel}
\usepackage[T1]{fontenc}
\usepackage{mathtools}
\usepackage{amsfonts}
\usepackage{fancyhdr}
\usepackage{amssymb}
\pagestyle{fancy}
\fancyhf{}
\rhead{Václav Luňák}
\lhead{ADS I - Internet}
\cfoot{\thepage}
\begin{document}
\part*{Internet}
\section{Popis řešení}
Vytvoříme graf následujícím způsobem:
\begin{itemize}
\item 
$n$ vrcholů budou tvořit jednotlivé domy
\item
jeden vrchol bude reprezentovat přímé připojení k internetu
\item
domy A a B budou spojeny hranou ohodnocenou $c(A, B)$
\item
dům A bude s internetem spojený hranou ohodnocenou $\mathcal{C}(A)$
\end{itemize}

Na takto vytvořeném grafu pak spustíme algoritmus na nalezení minimální kostry, např. Kruskalův. Optickým kabelem pak propojíme ty domy, které jsou spojeny hranou v minimální kostře a k internetu připojíme ty, jež jsou s ním v minimální kostře spojeny hranou. 
\section{Pseudokód}
\begin{verbatim}
A = null;
foreach (v in V) Make-Set(V);
foreach((u,v) in setříděné E podle vah)
   if (Set(u)!=Set(v)){
      A = A + (u,v);
      Union(u,v);
   }
return A;
\end{verbatim}
\section{Důkaz správnosti}
Z podstaty zadání můžeme předpokládat, že každá hrana má kladné ohodnocení. Minimální kostra tedy představuje souvislý podgraf s minimálním součtem ohodnocení. Takto vytvořený graf zároveň přesně odpovídá problému; každý dům lze spojit s internetem i se všemi ostatními domy a hledáme nejlevnější připojení všech domů.\\

Protože Kruskalův algoritmus nevyžaduje unikátní hrany, bude tento postup fungovat pro obě varianty zadání, přičemž pro variantu (a) budou z internetu vést hrany ohodnocené $\mathcal{C}$. 
\section{Časová složitost}
Použili jsme Kruskalův algoritmus na hledání minimální kostry, tedy složitost algoritmu je $O(m \log m)$.
\section{Prostorová složitost}
O každém vrcholu i hraně si pamatujeme konstantní množství informací, tedy prostorová složitost algoritmu je $O(n+m)$.
\end{document}