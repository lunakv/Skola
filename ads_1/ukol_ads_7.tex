\documentclass[11pt,a4paper]{article}
\usepackage[utf8]{inputenc}
\usepackage[czech]{babel}
\usepackage[T1]{fontenc}
\usepackage{amsmath}
\usepackage{amsfonts}
\usepackage{fancyhdr}
\usepackage{amssymb}
\pagestyle{fancy}
\fancyhf{}
\rhead{Václav Luňák}
\lhead{ADS I - Příklad 10b}
\cfoot{\thepage}
\begin{document}
\part*{Příklad 10b}
\section{Popis řešení}
Na nalezení kostry s minimálním součinem použijeme stejný algoritmus jako na nalezení běžné minimální kostry, například Kruskalův algoritmus.
\section{Pseudokód}
Identický jako v úkolu \textsc{Internet}.
\section{Důkaz správnosti}
Vytvořme nový graf, který bude stejný jako graf původní s rozdílem, že váhy hran nového grafu budou logaritmy vah z hran původních. Máme ostře kladné hrany, tedy logaritmus je vždy definován. Na takovémto grafu spustíme algoritmus pro hledání minimální kostry. Dostaneme kostru s ohodnocením $\sum_{i \in E(T)} \log i = \log \left( \prod_{i \in E(T)} i \right)$. \\

Logaritmus je rostoucí funkce, tedy minimalizovat logaritmus součinu nutně znamená minimalizovat součin. Odtud nám už vyplývá rovnost minimálních koster v obou grafech.\\

Jelikož jsme schopni takto dokázat, že se rovná minimální kostra v novém grafu a minimální kostra grafu původního (resp. množiny min. koster pro neunikátní ohodnocení), není třeba nový graf vytvářet a lze známý algoritmus spustit rovnou na původním grafu.
\section{Časová složitost}
Uvedený Kruskalův algoritmus má časovou složitost $O(m\log m)$.
\section{Prostorová složitost}
Prostorová složitost Kruskalova algoritmu je $O(n+m)$.
\end{document}