\documentclass{scrartcl}
\usepackage[utf8]{inputenc}
\usepackage{scrlayer-scrpage}
\usepackage[czech]{babel}
\usepackage{amsmath}
\usepackage{fontspec}

\setmainfont[Ligatures=TeX]{Linux Libertine O}
\setsansfont[Ligatures=TeX]{Linux Biolinum O}

\cohead{Foundations of Theoretical Cryptography}
\lohead{Úkol č. 4}
\rohead{Václav Luňák}

\DeclareMathOperator{\Prob}{\text{Pr}}
\DeclareMathOperator{\xor}{\oplus}

\begin{document}
\subsection*{Úloha č. 2}
Adversary se nejprve pošle šifrovacímu orákulu zprávu $(0^n||0^n)$. Feistelova síť se třemi rundami vrátí ciphertext
\begin{align*}
    E_k(0^n||0^n) &= (a||b), \begin{cases}
        a = F_2(F_1(0)), \\
        b = F_1(0) \xor F_3(a).
    \end{cases} 
\end{align*} 

Následně se adversary zeptá dešifrovacího orákula na inverz ciphertextu $(a||0^n)$. Pro Feistelovu síť obdrží
\begin{align*}
    D_k(a||0^n) &= (c||d), \begin{cases}
        c = F_3(a) \xor F_1(d), \\
        d = a \xor F_2(F_3(a)).
    \end{cases}
\end{align*}

Všimněme si, že $x := b \xor c = F_1(d) \xor F_1(0)$. Nakonec se adversary dotáže šifrovacího orákula na ciphertext zprávy $(x||d)$.
\begin{align*}
    E_k(x||d) &= (e||f), \begin{cases}
        e = d \xor F_2(x\xor F_1(d)) = d \xor F_2(F_1(0)) = d\xor a, \\
        f = x \xor F_1(d) \xor F_3(e).
    \end{cases}
\end{align*}

Vidíme tedy, že pro Feistelovu síť o třech rundách bude vždy platit $e = d\xor a$. Jelikož pro náhodnou permutaci toto bude platit s pravděpodobností $1/2^n$, máme účinného distinguishera těchto Feistelových sítí, což nám dává spor s tím, že jsou silně pseudonáhodné.
\end{document}