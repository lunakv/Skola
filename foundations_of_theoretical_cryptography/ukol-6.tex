\documentclass{scrartcl}
\usepackage[utf8]{inputenc}
\usepackage{scrlayer-scrpage}
\usepackage[czech]{babel}
\usepackage{amsmath}
\usepackage{fontspec}
\usepackage[activate={true,nocompatibility},final,tracking=true,factor=1100,stretch=10,shrink=10]{microtype}

\setmainfont[Ligatures=TeX]{Linux Libertine O}
\setsansfont[Ligatures=TeX]{Linux Biolinum O}

\cohead{Foundations of Theoretical Cryptography}
\lohead{Úkol č. 6}
\rohead{Václav Luňák}

\DeclareMathOperator{\Prob}{\text{Pr}}
\DeclareMathOperator{\xor}{\oplus}

\begin{document}
    \subsection*{CTR mód}
    Vytvoříme zprávy $m_0 = 0^n$ a $m_1 = 1^n$ a dostaneme zpět challenge ciphertext $(IV,c)$ pro zprávu $m_b$, kde $c=c_1c_2\dots c_l$. Dotážeme se dešifrovacího orákula na ciphertext $(IV+1,c')$ pro $c'=c_2c_3\dots c_lc_1$ a obdržíme zprávu $m'$.

    Z fungování CTR módu snadno vidíme, že pokud je $b = 0$, prvních $l-1$ bloků $m'$ bude nulových, jelikož
    \begin{align*}
        m'_i \xor F_k(IV+1+i) = c_{i+1} = {m_b}_{(i+1)} \xor F_k(IV+i+1)
    \end{align*}
    pro $i \in \{1,\dots,l-1\}$. Pokud je naopak $(IV,c)$ ciphertextem $m_1$, stejným argumentem bude prvních $l-1$ bloků $m'$ zaplněno jedničkami. Sestrojili jsme tedy adversaryho úspěšného v CCA experimentu s pravděpodobností 1, tudíž CTR mód není CCA-secure.

    \subsection*{OFB mód}
    Mějme zprávy a ciphertext stejně jako v předchozím případě. Po obdržení challenge ciphertextu se dotážeme dešifrovacího orákula na ciphertext $(IV,c')$, kde $c' = c_1c_2\dots c_{l-1}d$ pro nějaké $d \neq c_l$ a obdržíme zprávu $m'$. Jelikož šifrování a dešifrování zprávy v OFB módu můžeme zapsat jako
    \begin{align*}
        c_i &= m_i \xor F_k(F_k(\cdots F_k(IV))) \\
        m_i &= c_i \xor F_k(F_k(\cdots F_k(IV))),
    \end{align*}
    kde funkce $F_k$ je do sebe $i$-krát vnořená, vidíme, že hodnota (de)šifrovaného bloku není závislá na hodnotách ostatních bloků. Jelikož se navíc prvních $l-1$ bloků $c'$ a $c$ shodují (a používají stejnou $IV$), vyplývá z toho, že prvních $l-1$ bloků $m'$ se bude shodovat s prvními $l-1$ bloky $m_b$, čímž snadno určíme, která zpráva byla šifrována. Ani OFB mód tedy není CCA-secure.

    \subsection*{CBC mód}
    Mějme zprávy a ciphertext jako v předchozích případech. Zeptáme se dešifrovacího orákula na ciphertext $(IV,c')$, kde $c' = c_1c_2\dots c_{l-1}d$ pro nějaké $d \neq c_l$ a obdržíme zprávu $m'$. Protože dešifrování je definováno jako
    \begin{align*}
        m_i = F_k^{-1}(c_i) \xor c_{i-1},
    \end{align*}
    vidíme, že hodnota $i$-tého bloku dešifrované zprávy závisí pouze na $c_i$ a $c_{i-1}$, popř. $IV$ pro první blok. Tyto hodnoty jsou pro prvních $l-1$ bloků $c'$ a $c$ shodné, tudíž i prvních $l-1$ bloků $m'$ a $m_b$ musí být shodných. Z toho snadno identifikujeme $b$ a dokážeme tak, že CBC mód není CCA-secure.
\end{document}