\documentclass{scrartcl}
\usepackage[utf8]{inputenc}
\usepackage{scrlayer-scrpage}
\usepackage[czech]{babel}
\usepackage{amsmath}

\cohead{Foundations of Theoretical Cryptography}
\lohead{Úkol č. 3}
\rohead{Václav Luňák}

\DeclareMathOperator{\Prob}{\text{Pr}}

\begin{document}
    \subsection*{Úloha č. 1}
    \subsubsection*{a)}
    Předpokládejme, že $F'$ není pseudonáhodná funkce. Mějme distinguishera $D$ pro funkci $F$ s orákulem $O$ a distinguishera $D'$ pro $F'$. $D$ poté sestrojíme následujícím způsobem:
    \begin{itemize}
        \item Když $D'$ vygeneruje zprávu $m$, dotážeme se orákula na $0||m$ a $1||m$ a vrátíme $D'$ hodnotu $O(0||m)||O(1||m)$.
        \item $D$ má stejnou návratovou hodnotu $D'$
    \end{itemize}
    Z předpokladu existuje polynomiální $D'$ nezanedbatelně rozlišující $F(0||m)||F(1||m)$. Jelikož na každý dotaz $D'$ generuje $D$ právě dva dotazy, vytvořili jsme polynomiálního distinguishera nezanedbatelně rozlišujícího $F$, čímž dostáváme spor s její pseudonáhodností.
    \subsubsection*{b)}
    Mějme $x_0 = 0^{n-2}||1$ a $x_1 = 0^{n-1}$. Vidíme, že prvních $n$ bitů $F'_k(x_0)$ je shodných s posledními $n$ bity $F'_k(x_1)$, jelikož jsou obě hodnoty rovny $F_k(0^{n-1}||1)$. Pravděpodobnost, že tato rovnost platí u náhodné funkce, je $1/2^n$. Distinguisher tedy dokáže rozlišit $F'_k$ s nezanedbatelnou pravděpodobností $1 - 1/2^n$, tudíž $F'$ není pseudonáhodná funkce.

    \subsection*{Úloha č. 2}
    Nechť $F$ je pseudonáhodná funkce. Mějme následující schéma:
    \begin{itemize}
        \item $G(1^n)$ generuje náhodné $k,j$ délky $n$.
        \item $Enc_{j,k}(m)$ pro $n$-bitovou zprávu $m$ nejprve vygeneruje náhodné $n$-bitové číslo $r$. Poté, pokud $m = j$, vrátí zprávu $(0^{2n}||j)$, jinak vrátí $(r||m \oplus F_k(r)||j)$.
        \item $Dec_{j,k}(c)$ je přirozený inverz $Enc$.
    \end{itemize}

    Ze skript víme, že konstrukce typu $(r||m \oplus F_k(r))$ je CPA-bezpečná a tudíž má i indistinguishable multiple encryptions. Pokud si adversary nevybral za jednu ze zpráv $j$, nepřidává tento suffix žádnou rozlišovací hodnotu. Pravděpodobnost zvolení $j$ je pouze $2t/2^n$ a tudíž zanedbatelná.
    
    Adversary v CPA indistinguishability experimentu naproti tomu může nejprve od orákula snadno zjistit hodnotu $j$ zašifrováním libovolné zprávy a poté zašifrovat např. $m_0 = j$ a $m_1 = \text{not}(j)$. Výsledné $b'$ je pak rovno 0 právě tehdy, když $c = 0^{2n}||j$. Snadno nahlédneme, že tento adversary selže pouze v případě, že náhodné $r = 0^n$ a $F_k(r) = \text{not}(j)$, tedy se zanedbatelnou pravděpodobností $1/2^{2n}$. 

    Vytvořili jsme tedy schéma, které má indistinguishable multiple encryptions, ale není CPA-bezpečné.
\end{document}