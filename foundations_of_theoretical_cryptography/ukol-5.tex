\documentclass{scrartcl}
\usepackage[utf8]{inputenc}
\usepackage{scrlayer-scrpage}
\usepackage[czech]{babel}
\usepackage{amsmath}
\usepackage{fontspec}

\setmainfont[Ligatures=TeX]{Linux Libertine O}
\setsansfont[Ligatures=TeX]{Linux Biolinum O}

\cohead{Foundations of Theoretical Cryptography}
\lohead{Úkol č. 5}
\rohead{Václav Luňák}

\DeclareMathOperator{\Prob}{\text{Pr}}
\DeclareMathOperator{\xor}{\oplus}

\begin{document}
\subsection*{Úloha č. 1}
Zpráva se rozdělí na 8 bloků po 128 bitech. K samotné zprávě ještě musíme přidat jeden blok $IV$ a jelikož je délka zprávy přesným násobkem délky bloku, nejspíše i jeden blok paddingu. Výsledný ciphertext tedy má délku 10 bloků, neboli 1280 bitů. Délka klíče pro velikost zprávy nehraje roli.

\subsection*{Úloha č. 2}
Ukážeme, že pokud by existoval adversary $A$, pro nějž by tato pravděpodobnost nebyla zanedbatelná, dokážeme z něj sestrojit distinguishera $D$ pro $F$.

Mějme orákulum $O$. Distinguishera sestrojíme následovně:
\begin{enumerate}
    \item Spustíme $A(1^n)$.
    \item Když si $A$ vyžádá od šifrovacího orákula ciphertext zprávy $m$, vygenerujeme náhodné $IV \in \{0,1\}^n$ a zašifrujeme $m$ v CTR módu pomocí $O$ jako šifrovací funkce. Výsledný ciphertext vrátíme $A$.
    \item Když $A$ vygeneruje dvě zprávy $m_0, m_1$, zvolíme si náhodný bit $b \in \{0,1\}$ a vrátíme zprávu $m_b$ zašifrovanou stejným postupem jako výše.
    \item Odpovídáme na dotazy $A$ jako výše, dokud nedostaneme výsledný bit $b'$. Vrátíme 0 právě tehdy, když $b = b'$.
\end{enumerate}

Pokud je $O$ pseudonáhodná funkce, adversary se chová stejně jako v experimentu $\textsf{PrivK}^\textsf{cpa}_{A, \Pi}(n)$, a tudíž
\begin{align*}
    \Prob_{k\leftarrow \{0,1\}^n}\left[ D^{F_k(\cdot)}(1^n) = 1 \right] = \Prob\left[ \textsf{PrivK}^\textsf{cpa}_{A, \Pi}(n) = 1 \right],
\end{align*}
kde $k$ je zvoleno rovnoměrně náhodně. 

Pokud je naproti tomu $O$ zcela náhodná funkce, chová se $A$ stejně jako v experimentu $\textsf{PrivK}^\textsf{cpa}_{A, \tilde{\Pi}}(n)$, z čehož dostáváme
\begin{align*}
    \Prob_{f \leftarrow \textsf{Func}_n} \left[ D^{f(\cdot)}(1^n) = 1 \right] = \Prob\left[ \textsf{PrivK}^\textsf{cpa}_{A, \tilde{\Pi}}(n) = 1 \right],
\end{align*}
kde $f$ je zvolena rovnoměrně náhodně. Jelikož předpokládáme, že šifra není CPA-secure, bude pro nějakého adversary platit 
\begin{align}
    \left\vert \Prob\left[ \textsf{PrivK}^\textsf{cpa}_{A, \Pi}(n) = 1 \right] - \Prob\left[ \textsf{PrivK}^\textsf{cpa}_{A, \tilde{\Pi}}(n) = 1 \right] \right\vert > \textsf{negl}(n).
\end{align}
Spojením z rovnostmi výše pak dostáváme, že musí rovněž platit
$$
    \left\vert  \Prob_{k\leftarrow \{0,1\}^n}\left[ D^{F_k(\cdot)}(1^n) = 1 \right] - \Prob_{f \leftarrow \textsf{Func}_n} \left[ D^{f(\cdot)}(1^n) = 1 \right] \right\vert > \textsf{negl}(n).
$$
Tím jsme dostali spor s pseudonáhodností $F$ a tudíž dokázali, že nerovnice (1) nemůže platit.
\end{document}