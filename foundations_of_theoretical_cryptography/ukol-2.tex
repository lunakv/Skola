\documentclass{scrartcl}
\usepackage[utf8]{inputenc}
\usepackage{scrlayer-scrpage}
\usepackage[czech]{babel}
\usepackage{amsmath}
\usepackage{amsfonts}

\cohead{Foundations of Theoretical Cryptography}
\lohead{Úkol č. 2}
\rohead{Václav Luňák}

\DeclareMathOperator{\Prob}{\text{Pr}}
\DeclareMathOperator{\A}{\mathbb{A}}
\begin{document}
    \subsection*{Úloha č. 1}
    Mějme $q(n)$ polynomiální horní hranici na délku ciphertextu pro zprávy délky 1. Adversary $\A$ zvolí zprávy $m_0$ délky 1 a $m_1$ délky $q(n) + n$. Pokud má obdržený ciphertext $c$ délku větší než $q(n)$, ví $\A$, že $c = E(m_1)$, jinak hádá náhodně. Pravděpodobnost, že $\A$ uhádne zprávu, potom je
    \begin{align*}
        \Prob[b' = b] &= \Prob[b' = b\ \&\ |c| > q(n)] + \Prob[b' = b\ \&\ |c| \leq q(n)] \\
        &= \Prob[|c| > q(n)] + 1/2 \cdot (1 - \Prob[|c| > q(n)]) \\
        &= 1/2 + 1/2 \cdot \Prob[|c| > q(n)]
    \end{align*}
\end{document}