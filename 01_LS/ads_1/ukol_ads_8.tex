\documentclass[11pt,a4paper]{article}
\usepackage[utf8]{inputenc}
\usepackage[czech]{babel}
\usepackage[T1]{fontenc}
\usepackage{amsmath}
\usepackage{amsfonts}
\usepackage{fancyhdr}
\usepackage{amssymb}
\pagestyle{fancy}
\fancyhf{}
\rhead{Václav Luňák}
\lhead{ADS I - Rozdělení BVS}
\cfoot{\thepage}
\begin{document}
\part*{Rozdělení BVS}
\section{Popis řešení}
Vytvoříme rekurzivní funkci, která pro daný strom a danou hodnotu vrátí uspořádanou dvojici rozdělených stromů. Naše funkce začne v kořeni a bude z něj vyhledávat daný prvek. Iterace se pro vrchol $v$ bude skládat ze 3 kroků:
\begin{enumerate}
\item[1]
Zjistíme, ve kterém podstromu $v$ se nachází daná hodnota. Druhý podstrom obsahuje pouze prvky menší/větší než $k$, tedy se jím prozatím nemusíme dále zabývat.
\item[2]
Spustíme rozdělující funkci rekurzivně na podstrom, ve kterém je hodnota.
\item[3]
Když dostaneme výsledek z rekurzivního volání (dvojice $(a,b)$), nejprve změníme syna $v$ v prohledávaném směru za složku výsledku v opačném směru. (Pokud jsme prohledávali pravý podstrom, nastavíme jako pravého syna $v$ strom $a$.) Funkce pak vrátí dvojici $(v,b)$, pokud se v rekurzi prohledával pravý podstrom, respektive $(a, v)$, pokud se prohledával levý podstrom. 
\end{enumerate}
Rekurze se zastaví, když najdeme vrchol s hledanou hodnotu nebo když prohledávaný vrchol neexistuje (je roven null). V prvním případě vrátíme dvojici $(a,b)$, kdy $a$ je levý podstrom vrcholu a $b$ je vrchol a jeho pravý podstrom. V druhém případě vrátíme dvojici $(null, null)$.
\section{Pseudokód}
\begin{verbatim}
<node, node> Split(node root, int k){
   if (root == null) return <null, null>;
   if (root.val == k) return <root.L, root+root.R>;
   if (root.val > k){
      x = Split(root.L, k);
      root.L = x.2;
      return <x.1, root>;
   }
   x = Split(root.R, k);
   root.R = x.1;
   return <root, x.2>;
}
\end{verbatim}

\section{Důkaz správnosti}
Nejprve ověříme, že krajní případy rekurze vrací správné hodnoty. V situaci, kde jsme našli $k$, je navrácená dvojice stromů triviální.\\

Pokud jsme funkci zavolali na null, znamená to, že jsme se snažili v předchozím kroku hledat $k$ ve směru, ve kterém $v$ nemělo syna (BÚNO vpravo). Leví potomci $v$ i $v$ jsou tedy menší než $k$, tudíž po vrácení <null, null> z našeho volání se o řád výš správně vrátí <$v$, null>, jelikož strom vycházející z $v$ nemá žádné prvky $\geq k$. Pro druhou stranu probíhá zdůvodnění analogicky.\\

Nyní tedy můžeme přejít k indukčnímu kroku. BÚNO root.val < $k$. Zavoláním funkce na pravý podstrom dostaneme dvojici stromů obsahující prvky z tohoto podstromu a odpovídající zadání. Jelikož jsme použili prvky z pravého syna rootu, jsou všechny větší než root, tedy můžeme levý z dvojice vrácených stromů nastavit jako pravého syna a stále dostaneme validní BVS. \\

Protože navíc víme, že root.val < $k$, víme, že levý podstrom rootu obsahuje také pouze prvky menší než $k$. Když tedy přidáme levý z dvojice vrácených stromů jako pravého syna, dostaneme strom, ve kterém jsou všechny prvky menší než $k$.\\

Když tento strom použijeme jako levý do návratové dvojice a pravý strom zachováme z předchozí iterace, dostaneme dvojici stromů, která obsahuje root a všechny jeho potomky a splňuje zadání. Toto je přesně požadovaná návratová hodnota funkce.\\

Pro root.val > $k$ zdůvodníme stejně, pouze s opačnými směry. Indukční krok je tedy v pořádku a algoritmus tím pádem validní.
\section{Časová složitost}
Každá iterace provede krom rekurzivního volání pouze konstantní množství operací. Při rekurzivním volání se vždy dostaneme do o jedna nižší úrovně stromu. Časová složitost je tedy $O(výška stromu)$.
\section{Prostorová složitost}
Iterace si pamatují pouze konstantní množství dat a iterací je nejvýše $n$ pro úplně nevyvážený strom. Dále si pamatujeme konstantní množství dat ke každému vrcholu. Prostorová složitost algoritmu je tedy $O(n)$.
\end{document}