\documentclass[11pt,a4paper]{article}
\usepackage[utf8]{inputenc}
\usepackage[czech]{babel}
\usepackage[T1]{fontenc}
\usepackage{amsmath}
\usepackage{amsfonts}
\usepackage{fancyhdr}
\usepackage{amssymb}
\pagestyle{fancy}
\fancyhf{}
\rhead{Václav Luňák}
\lhead{ADS I - Lupiči}
\cfoot{\thepage}
\begin{document}
\part*{}
\section{Popis řešení}
K nalezení cesty použijeme modifikovaný Dijkstrův algoritmus. Pravděpodobnosti přepadení budeme chápat jako ceny hran a hledat budeme nejkratší cestu, ovšem při aktualizaci vrcholů nebudeme pro výpočet vzdálenosti od počátku hrany přičítat, ale počítat výslednou pravděpodobnost pomocí principu inkluze a exkluze. Pravděpodobnost přepadení na nejbezpečnější cestě získáme, když zpracujeme cíl. Pokud chceme najít nejen nejmenší pravděpodobnost, ale zároveň i cestu s touto pravděpodobností, algoritmus dále modifikujeme například tak, že každý vrchol si při přidání na prioritní frontu bude pamatovat i vrchol, který ho tam přidal.
\section{Pseudokód}
\begin{verbatim}
foreach v je vrchol
  d[v] = nekonečno
  pred[v] = null
d[start] = 0

while halda není prázdná {
  v = ExtractMin()
  if v == cil break
  foreach w soused od v
    alt = d[v] + vw - d[v]*vw
    if (d[w] > alt)
      d[w] = alt
      pred[w] = v
}

cesta = ""
v = cil
while v != start
  cesta = pred[v] + cesta
  v = pred[v]     
\end{verbatim}
\section{Důkaz správnosti}
Nejprve je potřeba zjistit, zda Dijkstrův algoritmus můžeme na takovýto graf použít. První podmínkou pro tento algoritmus je nezápornost všech ohodnocení. Tuto podmínku máme splněnou triviálně z definice pravděpodobnosti. Také si můžeme uvědomit, že žádný sled, který není cestou, nebude bezpečnější než cesta, tedy nikdy si nepomůžeme cyklem. Toto je taktéž zřejmé; pokud jsou všechny hrany cyklu ohodnoceny nulou, výsledná pravděpodobnost se nijak nezmění. Při jakýchkoliv nenulových hodnotách se pak nutně zvýší. Odtud a z faktu, že algoritmus vybírá vždy nejbližší vrchol, můžeme vyvodit, že při extrakci cíle může algoritmus skončit. \\

Je také nutno poukázat na to, že pokud známe pravděpodobnost přepadení do předposledního vrcholu nějaké cesty a ohodnocení poslední hrany, nemusíme počítat výslednou pravděpodobnost celou znovu pomocí principu inkluze a exkluze, ovšem dá se vyjádřit jednoduše pomocí výrazu $a+b-ab$. \\

Toto tvrzení dokážeme následovně. Z principu inkluze a exkluze víme, že pravděpodobnost přepadení na cestě o dvou hranách s pravděpodobnostmi $a$ a $b$ je rovna $a+b-ab$. Jelikož známe výslednou pravděpodobnost přepadení na prvních $n-1$ hranách, můžeme pro účely našeho výpočtu nahradit tyto hrany jedinou hranou ohodnocenou touto výslednou pravděpodobností. Poté už máme cestu složenou ze dvou hran a můžeme využít vzorec uvedený výše. \\

Jiný náhled na ospravedlnění použití Dijkstrova algoritmu je ten, že zatímco klasická varianta hledá minima z délek cest, tato verze hledá minima z celkových pravděpodobností cest (získaných pomocí PIE). 
\section{Časová složitost}
Použili jsme Dijkstrův algoritmus modifikovaný pouze o konstantní množství operací, tedy časová složitost algoritmu bude v závislosti na implementaci $O(n^2+m)$ (pro pole), $O((n+m)\log n)$ (pro binární haldu) nebo $O(m+n\log n)$ (pro Fibonacciho haldu).
\section{Prostorová složitost}
Pamatujeme si konstantní množství informací o každém vrcholu i o každé hraně. Celková prostorová složitost je tedy $O(n+m)$.
\end{document}