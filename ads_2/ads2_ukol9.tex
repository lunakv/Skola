\documentclass{scrartcl}
\usepackage[utf8]{inputenc}
\usepackage{scrlayer-scrpage}
\usepackage[czech]{babel}
\usepackage{amsmath}

\cohead[ADS 2]{ADS 2}
\lohead[$k$-barevnost]
        {$k$-barevnost}
\rohead[Václav Luňák (Vašek)]
        {Václav Luňák (Vašek)}
\pagestyle{plain.scrheadings}

\begin{document}
\section{Popis řešení}
Pro zadaný graf a $k$ si vytvoříme proměnné

\begin{itemize}
        \item $x_{i,j}$ pro $i = 1..n, j = 1..k$
\end{itemize}

a konstanty

\begin{itemize}
        \item $e_{i,j} = 1$ právě tehdy, když mezi $i$-tým a $j$-tým vrcholem vede hrana 
\end{itemize}

S těmito proměnnými pak sestavíme tyto klauzule:
\begin{enumerate}
        \item $(x_{i,1}$ v $x_{i,2}$ v $...$ v $x_{i,k})$ $\forall i = 1..n$
        \item $(\neg x_{i,j}$ v $\neg x_{i,m})$ $\forall i = 1..n$ $\forall j,m = 1..k$ 
        \item $(\neg e_{i,j}$ v $\neg x_{i, m}$ v $\neg x_{j,m})$ $\forall i,j = 1..n$ $\forall m = 1..k$
\end{enumerate}

Výsledná formule pak bude konjunkcí všech těchto klauzulí. Takováto formule je splnitelná právě tehdy, když bude původní graf obarvitelný pomocí $k$ barev, čímž je převod na $k$-SAT kompletní.

\section{Důkaz správnosti}
Aby byl graf obarvitelný pomocí $k$ barev, musí existovat takové obarvení, ve kterém je každý vrchol obarven právě jednou barvou a žádné dva sousední vrcholy nejsou obarveny barvou stejnou. Ukážeme, že když ... \\

Pakliže existuje uspokojující obarvení grafu, nastavíme $x_{i,j} = 1$ právě pokud $i$-tý vrchol je obarvený $j$-tou barvou a $e_{i,j} = 1$ právě když mezi $i$-tým a $j$-tým vrcholem vede hrana. \\

Klauzule 1. jsou splněny triviálně, protože každý vrchol nějakou barvu má. Klauzule 2. jsou taktéž splněné, protože $i$-tý vrchol obarvený $j$-tou barvou nemůže být obarvený $m$-tou barvou. Splněné jsou i klauzule 3., protože dvojice vrcholů buď nesdílí hranu (tedy $\neg e_{i,j}$), nebo alespoň jeden z nich nemá $m$-tou barvu. Z existence obarvení tedy plyne splnitelnost formule.\\

Naopak pokud nalezneme splňující ohodnocení formule, obarvíme $i$-tý vrchol $j$-tou barvou, pokud $x_{i,j} = 1$. Klauzule 1. nám zaručují, že každý vrchol je obarvený nějakou barvou. Z klauzulí 2. pak vyplývá, že žádný vrchol nemůže být obarvem více barvami (protože pokud $x_{i,j} = 1$, ze splnitelnosti musí platit $x_{i,m} = 0$). Z klauzulí 3. dostáváme, že každá dvojice vrcholů buď nemá společnou hranu, nebo nejsou obarveny stejnou barvou.\\

Ukázali jsme tedy, že se jedná o korektní obarvení a tím pádem i evivalenci mezi splnitelností formule a obarvitelností grafu. Každá klauzule má navíc délku nejvýše $k$. Jedná se tedy o korektní převod na $k$-SAT.

\section{Složitost}
Odhadněme nyní časovou složitost převodní funkce. Považujme $k$ za konstantu a označme $n$ počet vrcholů grafu Ohodnocení konstant $e_{i,j}$ zabere čas $O(n^2)$. Klauzule 1. vytvoříme v $O(n)$, klauzule 2. v $O(n \cdot k^2) = O(n)$ a klauzule 3. v $O(n^2 \cdot k) = O(n^2)$. Celý převod tedy zabere $O(n^2)$. Problém $k$-barevnosti grafů je tedy takto převoditelný na $k$-SAT.
\end{document}
